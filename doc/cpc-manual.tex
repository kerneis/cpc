\documentclass[a4paper]{report}

\title{The CPC manual}
\author{Juliusz Chroboczek, Gabriel Kerneis\\
{\tt <jch@pps.jussieu.fr>, <kerneis@pps.jussieu.fr>}}
\date{12 June 2009}

\begin{document}
\maketitle

\chapter{The CPC language} \label{chapter:language}

CPC is a programming language designed for any situation where
even-driven programming is suitable --- most notably, for writing
concurrent programs.  The semantics of CPC is defined as a
source-to-source translation from CPC into plain C using a
technique known as {\em translation into Continuation Passing Style\/}
(CPS) \cite{strachey:continuations, plotkin:call-by-lambda}.

The main abstraction provided by CPC is a {\em continuation}, roughly
corresponding to what other concurrent programming systems term a {\em
thread\/} or {\em lightweight process\/}\footnote{From the
programmer's point of view, the main difference is that a thread has
a long-term identity (a thread or process identifier) which makes it
possible to have constructs such as {\em join\/} or {\em kill}.
Continuations, on the other hand, are transient: after execution of
some code, the former continuation no longer exists, and a new
continuation has been created.}.

\paragraph{Structure of a CPC program}

Just like a plain C program, a CPC program is a set of functions.
Functions in a CPC program are partitioned into ``cps'' functions and
``native'' functions; a global constraint is that a cps function can
only ever be called by another cps function, never by a native
function.  The precise set of contexts where a cps function can be
called is defined in Sec.~\ref{sec:contexts}.

Intuitively, cps code is ``interruptible'': it is possible to
interrupt the flow of a block of cps code in order to pass control to
another piece of code or to wait for an event to happen.  Native code,
on the other hand, is ``atomic'': if a sequence of native code is
executed, it must be completed before anything else is allowed to run.

Technically, native function calls are executed by using the machine's
native stack.  Cps function calls, on the other hand, are executed by
using a lightweight stack-like structure known as a continuation.
This arrangement makes CPC context switches extremely fast; the
tradeoff is that a cps function call is an order of magnitude slower
than a native call.  Thus, computationally expensive code should be
implemented in native code whenever possible.

Execution of a CPC program starts at a native function called {\tt
  main}.  This function usually starts by registering a number of
continuations with the CPC runtime (using {\tt cpc\_spawn},
Section~\ref{sec:cooperating}), and then passes control to the CPC
runtime (by calling {\tt cpc\_main\_loop}, Section~\ref{sec:bootstrapping}).

\paragraph{Implementatin of CPC}

CPC is implemented as a source-to-source translation from the CPC
language, a conservative extension of C, into event-driven C code.
The resulting C code is linked against the {\em CPC scheduler}, a
modest event loop that handles scheduling of CPC continuations.

The CPC scheduler manipulates three data structures: a queue of
runnable continuations, a priority queue of sleeping continuations,
and a set of queues of continuations blocked on condition variables or
waiting for I/O.

\section{The CPC language}

CPC is a conservative extension of the 1999 edition of the C
programming language; thus, the syntax of CPC is defined as a set of
productions to be added to the grammar defined in the ISO C99 standard
\cite{iso:c99}.

In addition to the reserved words in C99, CPC reserves the words
{\tt cps}, {\tt cpc\_yield}, {\tt cpc\_done}, {\tt cpc\_spawn}, 
{\tt cpc\_wait}, {\tt cpc\_sleep},
{\tt cpc\_io\_wait}, {\tt cpc\_attach}, {\tt cpc\_detach} and
{\tt cpc\_detached}.

\subsection{CPS contexts} \label{sec:contexts}

Any instruction, declaration, or function definition in CPC can be in
{\em cps context\/} or in {\em native context}.  Cps context is
defined as follows:
\begin{itemize}
\item the body of a cps function is in cps context
  (Sec.~\ref{sec:cpc-functions};
\item the body of a {\tt cpc\_spawn} statement is in cps context
  (Sec.~\ref{sec:cooperating});
\item the body of a {\tt cpc\_detached} or {\tt cpc\_attached} statement is
  in cps context (Sec.~\ref{sec:native-threads}).
\end{itemize}
Any construct that is not in cps context is said to be in native
context.

\subsection{CPS functions} \label{sec:cpc-functions}

\[ \mbox{function-specifier} ::= \mathtt{cps} \]

Functions can be declared as being CPS-converted by adding {\tt cps}
to the list of functions specifiers.  The effect of such a declaration
is to put the body of the function in cps context, thus making it
possible to use most of the CPC features.

\[ \mbox{block-item} ::= \mbox{function-definition} \]

Functions can be defined within other functions, as in Algol-family
languages; the inner function can access the variables bound by the
outer one.  Only cps functions can be inner functions, and they must
be within other cps functions.

Free variables of inner functions are copies of the variables of the
enclosing function; thus, a change to the value of the free variable
is not visible in the enclosing function.

\subsection{Bootstrapping} \label{sec:bootstrapping}

\begin{verbatim}
    void cpc_main_loop(void);
\end{verbatim}

\paragraph{\tt cpc\_main\_loop} Since \verb|main| is a native
function, some means is necessary to pass control to cps code.  The
function \verb|cpc_main_loop| invokes the CPC scheduler; it returns
when all continuations have been exhausted (i.e.\ where there is
nothing more to do).

\subsection{CPC statements}

\[ \mbox{statement} ::= \mbox{cpc-statement} \]

CPC has a number of statements not present in the C language.

\subsection{Cooperating: yielding, spawning} \label{sec:cooperating}

\begin{eqnarray*}
\mbox{cpc-statement} & ::= & 
     \mathtt{cpc\_yield} \mathtt{;} \\
 &|& \mathtt{cpc\_done} \mathtt{;} \\
 &|& \mathtt{cpc\_spawn}\ \mbox{statement}
\end{eqnarray*}

\paragraph{\tt cpc\_yield} The {\tt cpc\_yield} statement causes the
current continuation to be suspended, and placed at the end of the
queue of runnable continuations.  Control is passed back to the CPC
main loop.  This statement is only allowed in cps context.

\paragraph{\tt cpc\_done} The {\tt cpc\_done} statement causes the
current continuation to be discarded, and control to be passed back to
the main CPC loop.  This statement is only allowed in cps context.

\paragraph{\tt cpc\_spawn} The {\tt cpc\_spawn} statement causes a new
continuation that executes the argument to {\tt cpc\_spawn} to be
created and placed at the end of the queue of runnable continuations.
Execution then proceeds after the {\tt cpc\_spawn} statement (control
is {\em not\/} passed back to the main CPC loop).  This statement is
valid in arbitrary context.

\subsection{Synchronisation: condition variables}

\[ \mbox{cpc-statement} ::=
   \mathtt{cpc\_wait} \mathtt{(} \mbox{expression} \mathtt{)} \mathtt{;} \]

\begin{verbatim}
    typedef struct cpc_condvar cpc_condvar;
    void cpc_signal(cpc_condvar *);
    void cpc_signal_all(cpc_condvar *);
\end{verbatim}

\paragraph{\tt cpc\_wait} The {\tt cpc\_wait} statement places the
current continuation on the list of continuations waiting on the
condition variable passed as argument to {\tt cpc\_wait}.  Control is
passed back to the CPC loop.  This statement is only valid in cps
context.

\paragraph{\tt cpc\_signal} The function {\tt cpc\_signal} causes the
first of the continuations waiting on the condition variable passed as
argument to be moved to the tail of the queue of runnable
continuations.  Execution proceeds at the instruction following the
call to {\tt cpc\_signal}.

\paragraph{\tt cpc\_signal\_all} The function {\tt cpc\_signal\_all}
causes all of the continuations waiting on the condition variabled
passed as argument to be moved to the tail of the queue of runnable
continuations.  This function guarantees that the continuations will
be run in the order in which they were suspended.

\subsection{Sleeping}

\[ \mbox{cpc-statement} ::=
     \mathtt{cpc\_sleep} \mathtt{(} \mbox{expression}
                                   [ \mathtt{,}\ \mbox{expression}
                                     [ \mathtt{,}\
                                         \mbox{expression} ]]
                         \mathtt{)} \mathtt{;} \]

\paragraph{cpc\_sleep} The statement {\tt cpc\_sleep} takes three
arguments: a time in seconds, a time in microseconds, and a condition
variable.  It causes the current continuation to be suspended until
either the specified amount of time has passed, or the condition
variable is signalled, whichever happens first.

The third argument can be omitted if no interruption is necessary.
The second argument can be omitted if sub-second accuracy is not
needed.

This statement is only valid in cps context.

\subsection{Waiting for I/O}
\[ \mbox{cpc-statement} ::=
   \mathtt{cpc\_io\_wait} \mathtt{(} \mbox{expression}
                                   \mathtt{,}\ \mbox{expression}
                                   [ \mathtt{,}\ \mbox{expression} ]
                         \mathtt{)} \mathtt{;}\]

\paragraph{cpc\_io\_wait} The statement {\tt cpc\_io\_wait} takes three
arguments: a file descriptor, a direction, and a condition variable.
The direction can be one of {\tt CPC\_IO\_IN}, meaning input, or {\tt
  CPC\_IO\_OUT}, meaning output.

This statement causes the current continuation to be suspended until
either the given file descriptor is available for I/O in the given
direction, or the given condition variable is signalled, whichever
happens first.

This statement is only valid in cps context

\subsection{Interaction with native threads} \label{sec:native-threads}
\begin{eqnarray*}
\mbox{cpc-statement} ::=
 &|& \mathtt{cpc\_detach} \mathtt{;} \\
 &|& \mathtt{cpc\_attach} \mathtt{;} \\
 &|& \mathtt{cpc\_detached}\ \mbox{statement} \\
 &|& \mathtt{cpc\_attached}\ \mbox{statement}
\end{eqnarray*}

A continuation can be scheduled to be run by a native thread; intuitively,
the continuation ``becomes'' a native thread.  When this happens, we say
that the continuation has been {\em detached\/} from the CPC scheduler.
The opposite operation is known as {\em attaching\/} a detached
continuation to the CPC scheduler.

\paragraph{cpc\_detach, cpc\_attach} The {\tt cpc\_detach} statement
detaches the current continuation; the following statements are executed in
a dedicated native thread.  The opposite operation is performed by {\tt
  cpc\_attach}, which causes the current continuation to be scheduled by
the CPC scheduler.

The {\tt cpc\_detach} and {\tt cpc\_attach} statements can only appear in
cpc context.

\paragraph{cpc\_detached, cpc\_attached} The body of a {\tt cpc\_detached}
statement is run detached : an implicit {\tt cpc\_detach} is executed upon
entering the body, and a {\tt cpc\_attach} is executed upon exiting.  The
{\tt cpc\_attached} construct is dual : a {\tt cpc\_attach} is executed
upon entry, and a {\tt cpc\_detach} is executed upon exit.

\section{Limitations and implementation notes}

Not all legal C code is allowable in CPC.  Some of the limitations
described below are fundamental to the implementation technique of
CPC; others are just artefacts of the current implementation, and will
be lifted in a future version.

\subsection{Fundamental limitations}

The use of the {\tt longjmp} library function, and its variants, is
not allowed in CPC code.

\subsection{Current limitations}

Old-style (``K\&R'') function definitions are not supported.

\subsection{Time complexity of CPC operations}

The current implementation of CPC implements all the CPS operations in
constant time, with the following exceptions:
\begin{itemize} 
\item at the end of every iteration of the main loop (running all the
  runnable continuations once), a {\tt select} system call is made; this
  call runs in time proportional to the number of the highest active file
  descriptor;
\item when a continuation is queued on two structures simultaneously
  (because of {\tt cpc\_sleep} or {\tt cpc\_io\_wait} with a non-null
  last argument), invoking it requires dequeueing it from the second
  queue, which takes linear time in the worst case;
\item the {\tt cpc\_sleep} instruction runs in worst-case time
  proportional to the number of currently sleeping continuations;
\end{itemize}

\chapter{The CPC library}

The functions in the CPC library are themselves written in CPC, using
only the primitives documented in Chapter~\ref{chapter:language}.

All the functions in the CPC core library are declared in the file
{\tt cpc-lib.h}.

\section{Barriers}

\begin{verbatim}
    typedef struct cpc_barrier cpc_barrier;

    cpc_barrier *cpc_barrier_get(int count);
    cps void cpc_barrier_await(cpc_barrier *barrier);
\end{verbatim}

A barrier is a synchronisation construct that allows a set of
continuations to be woken up at the same time.  A barrier is
conceptually a queue of continuations and a count of continuations
remaining to wait for.

\paragraph{\tt cpc\_barrier\_get} The function {\tt cpc\_barrier\_get}
returns a new barrier initialised to wait for {\tt count}
continuations.

\paragraph{\tt cpc\_barrier\_await} The function {\tt cpc\_barrier\_await}
causes the current continuation to wait on the barrier given in
argument.  This function first decrements the barrier's count; if the
count reaches zero, it wakes up all of the continuations waiting on
the barrier.  Otherwise, it suspends the current continuation.

The function {\tt cpc\_barrier\_await} guarantees that the
continuations are run in the order in which they were suspended.

\section{Input/Output}

\subsection{Setting up file descriptors}

\begin{verbatim}
    int cpc_setup_descriptor(int fd, int nonagle);
\end{verbatim}

The function \verb|cpc_setup_descriptor| sets up the file descriptor
\verb|fd| into non-blocking mode, making it suitable for use by the
CPC runtime.  If \verb|nonagle| is true (non-zero), the descriptor is
assumed to refer to a socket and has the Nagle algorithm disabled (the
socket option \verb|TCP_NODELAY| is set).

This function returns 1 in case of success, -1 in case of failure.

\subsection{Input/Output}
\begin{verbatim}
  cps int cpc_write(int fd, void *buf, size_t count);
  cps int cpc_write_timeout(int fd, void *buf, size_t count,
                            int secs, int micros);
  cps int cpc_read(int fd, void *buf, size_t count);
  cps int cpc_read_timeout(int fd, void *buf, size_t count,
                           int secs, int micros);
\end{verbatim}

The functions \verb|cpc_write| and \verb|cpc_read| are CPC's versions
of the \verb|write| and \verb|read| system calls.  They return the
number of octets read/written in case of success; in case of failure,
they return -1 with \verb|errno| set.

The versions with \verb|timeout| appended return after \verb|secs|
seconds and \verb|micros| micreseconds if no I/O has been possible.
In this case, they return -1 with \verb|errno| set to \verb|EAGAIN|.

\begin{thebibliography}{MTHM90}

\bibitem[ISO99]{iso:c99}
Information technology --- programming language {C}.
International standard ISO/IEC~9899:1999, 1999.

\bibitem[Plo75]{plotkin:call-by-lambda}
G.~D. Plotkin.
Call-by-name, call-by-value and the lambda-calculus.
{\em Theoretical Computer Science}, 1:125--159, 1975.
Also published as Memorandum SAI--RM--6, School of Artificial
  Intelligence, University of Edinburgh, Edinburgh, 1973.

\bibitem[SW74]{strachey:continuations}
Christopher Strachey and Christopher P. Wadsworth.
Continuations: A mathematical semantics for handling full jumps.
Technical Monograph PRG--11, Oxford University Computing
Laboratory, Programming Research Group, Oxford, England, 1974.

\end{thebibliography}

\end{document}
